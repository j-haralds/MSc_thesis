\section{Background}
With an ever growing demand for convenient energy storage, the use of \glspl{lib} in portable devices have skyrocketed over the last 30 years. 



\gls{pinn}, \gls{ev}, \gls{li}

\textbf{Contents}
\begin{itemize}
    \item Why batteries? (Nobel prize 2019, a game changer in our modern lives, necessary for climate change)
    \item However, there are still problems, and the hunt for even more energy efficient batteries requires models to describe the physical properties of batteries. 
    \item While models based on \gls{fe} can be used to simulate such systems to a high accuracy, they are often computationally costly\cite{Asheri}. Au contraire, data-driven models such as \glspl{nn} are generally computationally cheap once trained, but could require large dataset to reach a satisfying accuracy. In addition, a purely data-driven approach fails to encompass prior knowledge of the underlying physics. 
    \item To address this, \gls{pinn} offers a middle-ground approach, where  physical knowledge of the system can be combined with the data-driven approach of \glspl{nn}. Specifically, in \glspl{pinn} solutions outside of the physically viable domain are heavily penalized, hopefully forcing the \gls{nn} outputs to be close to the true solution. Although first introduced during the 90s, \glspl{pinn} have gained popularity with the rise of modern computational power. 
\end{itemize}
\section{Aim and Scope}
